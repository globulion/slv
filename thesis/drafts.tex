%
\begin{equation}
\langle \Psi_0^{(1)} \vert {\BM \mu} \vert \Psi_0^{(1)} \rangle \cong 
\langle \Psi_0^{(0)} \vert {\BM \mu} \vert \Psi_0^{(0)} \rangle
+
\langle \delta \Psi_0^{(1)} \vert {\BM \mu} \vert \Psi_0^{(0)} \rangle
+
\langle \Psi_0^{(0)} \vert {\BM \mu} \vert \delta \Psi_0^{(1)} \rangle
\end{equation}
%
where $\vert \Psi_0^{(n)} \rangle$ denotes the ground state in $n$th order
and 
%
\begin{equation}
\vert \delta \Psi_0^{(1)} \rangle = \sum_S{^{'}} \frac{
\vert S \rangle \langle S \vert \mathscr{H}' \vert \Psi_0 \rangle 
}{E^{(0)}_0 - E^{(0)}_S}
\end{equation}
% 
for composite states with joint quantum number S as described in Eq.~\eqref{eq:general_state_vibr}.
It is straightforward to show that
%
\begin{equation}
\langle \Psi_0^{(0)} \vert {\BM \mu} \vert \Psi_0^{(0)} \rangle \cong 
{\BM \mu}_0 + \frac{\hbar}{4}\sum_i \frac{1}{M_i\omega_i} 
             \frac{\partial^2 {\BM \mu}_0 }{\partial Q_i^2} \Bigg|_{{\bf Q}_{0_A}}
\end{equation}
%
in which we used the Taylor expansion of the dipole operator
%
\begin{equation}
{\BM \mu} \cong {\BM \mu}_0 + \sum_i \frac{\partial {\BM \mu}_0 }{\partial Q_i} \Bigg|_{{\bf Q}_{0_A}} Q_i
 + \frac{1}{2} \sum_{ij} \frac{\partial^2 {\BM \mu}_0 }{\partial Q_i \partial Q_j} \Bigg|_{{\bf Q}_{0_A}} Q_i Q_j
\end{equation}
%
Next, we have
%
\begin{equation}
\langle \delta \Psi_0^{(1)} \vert {\BM \mu} \vert \Psi_0^{(0)} \rangle
+
\langle \Psi_0^{(0)} \vert {\BM \mu} \vert \delta \Psi_0^{(1)} \rangle =
2
 \sum_S{^{'}} \frac{
\langle \Psi_0 \vert {\BM \mu} \vert S \rangle \langle S \vert \mathscr{H}' \vert \Psi_0 \rangle 
}{E^{(0)}_0 - E^{(0)}_S}
\end{equation}
%



In this Section we will also derive the solvation-induced transition dipole
change by using RSPT approach.
In the first-order with respect to perturbation $\mathscr{H}'$ in Eq.~\eqref{e:RSPT-Hamiltonian},
the transition dipole related to the fundamental transition of $j$th mode 
can be written as
%
\begin{equation}
{\BM \mu}_j^{1 \leftarrow 0} \equiv \langle \Psi_{j_1}^{(1)} \vert \hat{\BM \mu} \vert \Psi_{j_0}^{(1)} \rangle 
\cong
%
\langle \Psi_{j_1} \vert \hat{\BM \mu} \vert \Psi_{j_0} \rangle
+
\langle \delta \Psi_{j_1}^{(1)} \vert \hat{\BM \mu} \vert \Psi_{j_0} \rangle
+
\langle \Psi_{j_1} \vert \hat{\BM \mu} \vert \delta \Psi_{j_1}^{(1)} \rangle
\end{equation}
%
where $\vert \Psi_{j_n}^{(1)} \rangle$ denotes the $\vert \Psi_{j_n} \rangle$ state in first order,
$\vert \Psi_{j_n} \rangle$ refers to the non\hyp{}perturbed eigenstate
and 
%
\begin{equation}
\vert \delta \Psi_{j_n}^{(1)} \rangle = \sum_S{^{'}} \frac{
\vert S \rangle \langle S \vert \mathscr{H}' \vert \Psi_{j_n} \rangle 
}{E^{(0)}_{j_n} - E^{(0)}_S}
\end{equation}
% 
for composite states with joint quantum number S as described in Eq.~\eqref{eq:general_state_vibr}.



% -===================
It was shown that, under varying electrostatic potential,
the change in the transition dipole is\citep{Cho.JCP.2009}
%
\begin{equation} \label{e:trans-dip-cho-pot}
 \frac{\partial{\BM \mu}(\phi)}{\partial\overline {\bf Q}_j} \Bigg|_{\overline{\bf Q}_{0_A}} -  
 \frac{\partial{\BM \mu}(\phi=0)}{\partial {\bf Q}_j} \Bigg|_{{\bf Q}_{0_A}} \cong
%
2\int_V {\rm d}\;{\bf r} \int_V    {\rm d}\;{\bf r}' \; 
\left\{
\frac{\partial \kappa({\bf r},{\bf r}')}{\partial Q_j} \Bigg|_{{\bf Q}_{0_A}} {\bf r} \phi({\bf r}')
\right\}
%
- \sum_i \frac{\delta Q_i}{M_i\omega_i^2}
\end{equation}
%
where $\kappa({\bf r},{\bf r}')$ denotes the charge response kernel
of solute and $\delta Q_i$ is given in Eq.~\eqref{e:struct-dist-cho} with
the interaction potential approximated by multipole expansion from
Eq.~\eqref{e:dmtp}. In a limiting case when $\nabla \phi$ is constant
(i.e., $\phi({\bf r}) = {\bf r} \cdot {\bf F} + {\rm const.}$), 
Eq.~\eqref{e:trans-dip-cho-pot} can be recast 
in a simple form
%
\begin{equation} \label{e:trans-dip-cho-field}
 \frac{\partial{\BM \mu}({\bf F})}{\partial\overline {\bf Q}_j} \Bigg|_{\overline{\bf Q}_{0_A}} -  
 \frac{\partial{\BM \mu}({\bf F}={\bf 0})}{\partial {\bf Q}_j} \Bigg|_{{\bf Q}_{0_A}} \cong
%
\left[
\frac{\partial {\BM \alpha}_0}{\partial Q_j} \Bigg|_{{\bf Q}_{0_A}} 
+ 
\sum_i \frac{1}{M_i\omega_i^2} 
\frac{\partial^2 {\BM \mu}_0}{\partial Q_j \partial Q_i} \Bigg|_{{\bf Q}_{0_A}}
\otimes
\frac{\partial {\BM \mu}_0}{\partial Q_i} \Bigg|_{{\bf Q}_{0_A}}
\right] 
\cdot {\bf F}
\end{equation}
%
where ${\BM \mu}_0$ and ${\BM \alpha}_0$ denote the solute's gas\hyp{}phase dipole moment
and polarizability, respectively.
