\chapter{Future}

In this Thesis, we have focused primarily on changes of the vibrational
properties of solute molecules upon solvation. In this Chapter, 
we provide the scope of future perspectives of {\sc Solvshift}
as well as the underlying theory of electronic and magnetic solvatochromism.

\section{Electronic Solvatochromism}

Let us consider the change in the electronic energy levels
that occurs when solute molecule becomes solvated by $N$
solvent molecules. Let $\mathscr{H}_A$ denote the Hamiltonian
of isolated solute molecule with the associated 
eigenstates and eigenenergies according to the time\hyp{}independent
Schr{\:o}dinger equation:
%
\begin{subequations}
\begin{align}
\mathscr{H}_A \vert A_n \rangle &=  E^A_n \vert A_n \rangle \\
\mathscr{H}_B^{\zeta} \vert B_m^{\zeta} \rangle &=  E^{\zeta}_m \vert B_m \rangle
\end{align}
\end{subequations}
%
The interaction Hamiltonian reads
%
\begin{equation}
\mathscr{H}' = \sum_{\zeta} \mathscr{V}_{AB}^{\zeta} + \frac{1}{2} \sum_{\zeta\eta} \mathscr{V}_{BB}^{\zeta\eta}
\end{equation}
%
The change of the $n\leftarrow 0$ transition energy
due to solvation can be represented as
%
\begin{equation}
\Delta E^A_{n0}({\bf \overline{Q}}_A, U) = E^A_n({\bf \overline{Q}}_A, U) - E^A_0({\bf \overline{Q}}_A, U)
\end{equation}
%
where ${\bf \overline{Q}}_A$ denotes the solute's structure
in ground state normal mode representation and 
$U$ is the solute\hyp{}solvent interaction potential.
It is customary to relate the above energy change as a function of the 
gas\hyp{}phase molecular properties, rather than the properties 
in solution. Therefore, let us Taylor\hyp{}expand the energy change
around the solute's gas\hyp{}phase structure
%
\begin{multline}
\Delta E^A_{n0}({\bf \overline{Q}}_A, U) \cong \Delta E^A_{n0}({\bf Q}_A, U) + 
\sum_i  \fderiv{\Delta E^A_{n0}({\bf Q}_A, U)}{Q_i} 
\left( \overline{Q}_i - Q_i \right) 
\\
 \approx \Delta E^A_{n0}({\bf Q}_A, U) + 
\sum_i \frac{1}{M_i\omega_i^2} \fderiv{\Delta E^A_{n0}({\bf Q}_A, U)}{Q_i} \fderiv{U}{Q_i}
\end{multline}
%
where $M_i$ and $\omega_i$ denote the reduced mass and harmonic
frequency of solute in electronic ground state. Here we assume that
all the derivatives are evaluated at ${\bf {Q}}_A$ (gas\hyp{}phase).
Note that in the above expression we have not expanded with respect to
$U$ because the corresponding derivatives are very difficult to deal with.
To treat the effects of $U$ we shall use perturbation theory.

\subsection{Long-range approximation}

If the wavefunction overlap can be neglected, the electronic state
can be written as Hartree product of separate molecular eigenstates.

Before we consider the full many\hyp{}body ensemble, for the first approximation
let us consider the case when there is only
one solvent molecule, i.e., $\mathscr{H}'=\mathscr{V}_{AB}$.
From second\hyp{}order perturbation theory, the energy of $n$th electronic level
is given by
%
\begin{equation}
E_n = \langle A_nB_0 \vert \mathscr{H}' \vert A_nB_0 \rangle 
+ \sum_{k\ne n} \sum_{m\ne 0} \frac{
\lvert \langle A_nB_0 \vert \mathscr{H}' \vert A_kB_m \rangle  \rvert^2
}{E^A_{nk} + E^B_{0m}}
\end{equation}
%
It is straightforward to see that, in the first order, 
the energy change due to solvation when solute is in gas\hyp{}phase
structure is given by
%
\begin{multline}
\delta^{(1)} E^A_{n0}({\bf Q}_A, U) = \sum_{x} \Bigg\{ 
\left( q^A_{nn;x} - q^A_{00;x} \right) \phi({\bf r_x}) \\
+ \left( {\BM \mu}^A_{nn;x} - {\BM \mu}^A_{00;x} \right) \cdot \nabla \phi({\bf r_x})  
+ \frac{1}{3} \left( {\BM \Theta}^A_{nn;x} - {\BM \Theta}^A_{00;x} \right) \cdot \nabla \phi({\bf r_x}) 
+ \ldots
\Bigg\}
\end{multline}
%
where we multipole\hyp{}expanded the solute\hyp{}solvent interaction 
potential. For example, $q^A_{nn;x}$ denotes the monopole moment located at ${\bf r_x}$
that refers to the $n$th electronic state charge density distribution (nuclei + electrons).
Here, we assume first the vertical transition so that we can use the difference multipoles.
If the structure upon excitation changes, one can use different positions 
of distributed multipoles in both electronic states.
The total first\hyp{}order effect, including the structural distortion due to solute-solvent
interaction, can be expressed as
%
\begin{multline}
\delta^{(1)} E^A_{n0}({\bf \overline{Q}}_A, U) \cong \sum_{x} \Bigg\{ 
\left( q^A_{nn;x} - q^A_{00;x} \right) \phi({\bf r_x}) 
+ \left( {\BM \mu}^A_{nn;x} - {\BM \mu}^A_{00;x} \right) \cdot \nabla \phi({\bf r_x})  \\
+ \frac{1}{3} \left( {\BM \Theta}^A_{nn;x} - {\BM \Theta}^A_{00;x} \right) \cdot \nabla \phi({\bf r_x}) 
+ \ldots
\Bigg\} \\
%
+
%
\sum_{x} \sum_i \frac{1}{M_i\omega_i^2} \fderiv{U}{Q_i}
\Bigg\{ 
\fderiv{ q^A_{nn;x} }{Q_i} \phi({\bf r_x}) 
+ \fderiv{  {\BM \mu}^A_{nn;x}}{Q_i} \cdot \nabla \phi({\bf r_x})  
+ \frac{1}{3} \fderiv{ {\BM \Theta}^A_{nn;x} }{Q_i} \cdot \nabla \phi({\bf r_x}) 
+ \ldots
\Bigg\}
\end{multline}
%
in which we neglected the derivatives of electrostatic potential since they are 
a minor contribution.

Now, let us analyze the second order energy shift. This can also
be expanded as conventional interaction energy with making use
of static and dynamic imaginary frequency dependent polarizabilities.
However, such approach is not convenient because it is difficult
to compute the excited state polarizability by using distributed
site approach. Note that we aim in studying rather large molecules
for which single polarizable point is definitely not sufficient to
describe induction and dispersion effects.

Therefore, we try to use different approach and keep using the 
sum\hyp{}over\hyp{}states expression. Mainly,
%
\begin{equation}
\delta^{(2)} E^A_{n0}({\bf Q}_A, U) = 
\sum_{m\ne 0} 
\left[
\sum_{k\ne n}  \frac{ V_{n0,km}^2 }{E^A_{nk} + E^B_{0m}}
%
- \sum_{k\ne 0} \frac{
V_{n0,km}^2
}{E^A_{nk} + E^B_{0m}}
\right]
\end{equation}
%
where $V_{nk,0m}$ denotes the interaction between the 
solute's $k\leftarrow n$ transition density with
solvent's $m\leftarrow 0$ transition density. In principle,
such densities can be also multipole expanded (exchange effects should
be small compared to the electrostatic effects).
The price we have to pay is to deal with the sum that do not need to
converge fast. Here we hope that due to the increased number
of nodes of the excited state wave functions, the multipole
interactions will tend to cancel each other when raising the
quantum numbers in the summations. This is because
the integration is performed over the whole space, not
just over one molecule (like in the SOS evaluation of polarizability which
has significantly slow convergence).
If one could obtain a reasonable estimate of this second order 
effect by making use of transition multipoles (e.g. evaluated by TrCAMM) 
referring to the solute and solvent electronic
states, the model could be tested. One might also take into account
electronic states that has substantial oscillator strength. Note that
many electronic transitions have oscillator strengths that are
close to zero suggesting that
many terms in the above summations could be ignored.

The total second order energy shift is thus given by
%
\begin{multline}
\delta^{(2)} E^A_{n0}({\bf \overline{Q}}_A, U) \approx 
\sum_{m\ne 0} ^{m_{\rm Max}}
\left[
\sum_{k\ne n} ^{k_{\rm Max}} \frac{ V_{n0,km}^2 }{E^A_{nk} + E^B_{0m}}
%
- \sum_{k\ne 0} ^{k_{\rm Max}} \frac{
V_{n0,km}^2
}{E^A_{nk} + E^B_{0m}}
\right] \\
+
\sum_i \frac{1}{M_i\omega_i^2} \fderiv{U}{Q_i}
\sum_{m\ne 0} ^{m_{\rm Max}}
\Bigg\{
  2 \left[ 
\sum_{k\ne n} ^{k_{\rm Max}} \frac{ V_{n0,km} }{E^A_{nk} + E^B_{0m}} \fderiv{V_{n0,km} }{Q_i}
-
\sum_{k\ne 0} ^{k_{\rm Max}} \frac{ V_{n0,km} }{E^A_{nk} + E^B_{0m}} \fderiv{V_{n0,km} }{Q_i}
\right] \\
+
\left[
\sum_{k\ne n} ^{k_{\rm Max}} \ln (E^A_{nk} + E^B_{0m}) \fderiv{E^A_{nk}}{Q_i} V_{n0,km}^2
-
\sum_{k\ne 0} ^{k_{\rm Max}}  \ln (E^A_{nk} + E^B_{0m}) \fderiv{E^A_{nk}}{Q_i} V_{n0,km}^2
\right]
\Bigg\}
\end{multline}
%
Here, the structural distortion contribution
which is quite formidable, however, is straightforward. 
This contribution contains the energy and 
transition multipole (TrCAMM) first derivatives of the solute molecule with respect to
ground state vibrational coordinate. They can be evaluated
numerically without any problems. However, the tests need to be done in the future
to determine whether it is possible to choose ${m_{\rm Max}}$ and ${k_{\rm Max}}$
such that the sums are converged to a reasonable degree. 
If the fully electrostatic approximation is employed, 
the computation of first and second order energy shift 
is very fast and the time consumption can be considered as negligible.
Exchange\hyp{}repulsion, induction and dispersion effects
affect the energy shift through the vibrational forces
due to solvent and were already derived before
inthe SolEFP model of vibrational solvatochromism. 
Probably, those non\hyp{}Coulombic effects should be included
in the accurate energy shift calculation.



\subsection{Time-dependent response formalism}

Let us consider the dynamics of the electronic excitation 
by examining the time evolution of nuclear configuration and
electronic eigenstates (and spectrum of energies) when solvated molecule absorbs one photon
of electromagnetic energy. It is particularly important because
such phenomena can be studied by ultrafast time-resolved spectroscopy.
To be added later.

