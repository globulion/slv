\documentclass{exam}
\printanswers
\usepackage{color}
\shadedsolutions
\renewcommand{\solutiontitle}{\noindent\textbf{Answer:}\enspace}
\definecolor{SolutionColor}{rgb}{0.8,0.9,1}
%\framedsolutions

% graphics
\usepackage{graphicx}
\usepackage{wrapfig}
\usepackage{lipsum}
\usepackage{afterpage}

% tables
\usepackage{dcolumn}
\usepackage{multirow}
\usepackage{hhline}
\usepackage{booktabs}
%\setlength{\tabcolsep}{0.5em} % for the horizontal padding
%{\renewcommand{\arraystretch}{2.2}% for the vertical padding

% mathematics
\usepackage{amsmath}
\usepackage{amsfonts}
\usepackage{amssymb}
\usepackage{amsbsy}
\usepackage{bm}
\usepackage{mathrsfs}
\usepackage{upgreek}
\allowdisplaybreaks

% symbols
\usepackage{gensymb}
\usepackage{textcomp}

% footnotes
\usepackage[symbol]{footmisc}

% caption fonts
\usepackage[font=scriptsize,labelfont=bf]{caption}

% boxed multiline equations
\usepackage[overload]{empheq}

\newcommand*{\widebox}[2][0.5em]{\fbox{\hspace{#1}$\displaystyle #2$\hspace{#1}}}
\usepackage{xpatch}%
\makeatletter
\xpatchcmd{\@Aboxed}{%
\boxed {#1#2}}
{%
\color{HotPink3}\boxed {\color{textcolor}#1#2}}
{}{}
\makeatother

%---------------------------------------------------
% SHORTCUTS
\newcolumntype{,}{D{.}{,}{2}}
\newcommand{\citee}[1]{\ensuremath{\scriptsize^{\citenum{#1}}}}
\newcommand{\HRule}{\rule{\linewidth}{0.2mm}}
% Quantum notation
\newcommand{\bra}[1]{\ensuremath{\bigl\langle {#1} \bigl\lvert}}
\newcommand{\ket}[1]{\ensuremath{\bigr\rvert {#1} \bigr\rangle}}
\newcommand{\braket}[2]{\ensuremath{\bigl\langle {#1} \bigl\lvert {#2} \bigr\rangle}}
\newcommand{\tbraket}[3]{\ensuremath{\bigl\langle {#1} \bigl\lvert {#2} \bigl\lvert {#3} \bigr\rangle}}
% Math
\newcommand{\pd}{\ensuremath{\partial}}
\newcommand{\DR}{\ensuremath{{\rm d} {\bf r}}}
%\newcommand{\BM}[1]{\ensuremath{\mbox{\boldmath${#1}$}}}
\newcommand{\BM}[1]{\bm{#1}}
% Chemistry (formulas)
\newcommand{\ch}[2]{\ensuremath{\mathrm{#1}_{#2}}}
% Math 
\newcommand{\VEC}[1]{\ensuremath{\mathrm{\mathbf{#1}}}}
% vector nabla
\newcommand{\Nabla}{\ensuremath{ \BM{\nabla}}}
% derivative
\newcommand{\FDer}[3]{\ensuremath{
\bigg(
\frac{\partial #1}{\partial #2}
\bigg)_{#3}}}
% diagonal second derivative
\newcommand{\SDer}[3]{\ensuremath{
\biggl(
\frac{\partial^2 #1}{\partial #2^2}
\biggr)_{#3}}}
% off-diagonal second derivative
\newcommand{\SSDer}[4]{\ensuremath{
\biggl(
\frac{\partial^2 #1}{\partial #2 \partial #3}
\biggr)_{#4}}}
% derivatives without bound
% derivative
\newcommand{\fderiv}[2]{\ensuremath{
\frac{\partial #1}{\partial #2}}}
% diagonal second derivative
\newcommand{\sderiv}[2]{\ensuremath{
\frac{\partial^2 #1}{\partial #2^2}
}}
% off-diagonal second derivative
\newcommand{\sderivd}[3]{\ensuremath{
\frac{\partial^2 #1}{\partial #2 \partial #3}
}}
% derivatives for tables
\newcommand{\fderivm}[2]{\ensuremath{
{\partial #1}/{\partial #2}}}
% diagonal second derivative
\newcommand{\sderivm}[2]{\ensuremath{
{\partial^2 #1}/{\partial #2^2}
}}
% off-diagonal second derivative
\newcommand{\sderivdm}[3]{\ensuremath{
{\partial^2 #1}/{\partial #2 \partial #3}
}}

%############################################################################
\begin{document}

\section{Isocontour maps}
\begin{questions}

%
\question The scribbled memo in Figure S3 that was written with your explanation on May 19 is as follows:
"map of derivatives in each point in space (because?) in 3D".
Could you explain the meaning of this memo again for me?

\begin{solution}

We analyze the quantity $\fderiv{\rho({\bf r})}{Q_{\rm XY}}$ which is a scalar function of three variables: 
$\left\{ x, y, z \right\}= {\bf r}$
stored in a 3D vector. Thus, $\fderiv{\rho({\bf r})}{Q_{\rm XY}}$
is a 3D-distribution of scalars. Also, those scalars can be positive or negative.
It is difficult to visualize such a function in a conventional way since we needed 4 dimensions to do that
(3 dimensions for arguments and 4th dimension for the value of the derivative).
Thus, the maps presented in the paper (Figures 7 and S3) are \emph{isocontour plots}, i.e., we 
plot the surface in 3D (domain of arguments - each point in space) that corresponds to the $\pm f$, where
$f$ is called the \emph{isovalue}. By plotting several isocontour plots for many isovalues we can imagine
the function $\fderiv{\rho({\bf r})}{Q_{\rm XY}}$. In other words, the fourth dimension is the isovalue.
\end{solution}

\end{questions}

\section{6-311++G** basis set}
\begin{questions}

%
\question Is "diffuse" for ++ alternatively said to be "exchange"?

\begin{solution}

No. "Diffuse" means that the basis function is spread over relatively large extent of space.
In other words, the orbital exponents $\alpha$ are very small. Recall the general form of GTO
to be $\phi({\bf r}) = (x-x_0)^l(y-y_0)^m(z-z_0)^n \times e^{-\alpha \vert {\bf r} - {\bf r_0}\vert ^2}$.
"++" denotes that diffuse functions are placed on hydrogens and heavy atoms.

\end{solution}

%
\question How can "polarization" be presented in wfn? By mixing (hybrizing) $s$ and $p$ atomic orbitals or other way?

\begin{solution}
"Polarization" in this case means just adding orbitals with higher angular quantum numbers compared to the smallest possible basis
set should have. For example, for carbon atom, the minimal basis set should consist of 1s, 2s and 2p orbitals. 
However, we might add also 3d (and other) orbitals. Those 3d (and f, g, h, ...) orbitals are called "polarization"
functions. Hybridization is another thing, not associated with the term "polarization basis function". 
\end{solution}

\question Could you explain 6-311 by writing wfns explicitly for carbon? This is because I can not understand the meaning of 6GTO for core and 12 orbitals for valence in the memo.

\begin{solution}
The labeling convention is "N-MG", where N is the core shell and M is valence shell. For example, 6-31G.
6-311G consists of 1s, 2s and 2p orbitals. You don't need to bother about these "6" and "311". They are not
crucial to understand the paper. Just to explain it, X-XXXG basis set denotes that each core basis function
is composed of X GTOs. In 6-311G basis, it refers to 6 GTOs forming one basis function for core electrons (in this case
only 1s orbital for carbon). XXX means that instead of using one basis function for a particular valence shell
we use three independent functions. For example, for carbon valence shell is 2s and 2p. So, we shall be using three
independent 2s orbitals, 3 independent 2p$_x$ orbitals etc. Now, 311 means that the first set of these functions
is composed of 3 GTOs. The second and third sets are composed of only one GTO.
\end{solution}

\end{questions}


\section{Other questions}
\begin{questions}

\question Could you explain why Car*Cbt (approx of Car/bt) contains 8 terms in coupled cluster?

\begin{solution}
Actually, it has two terms. I told about 8 because in CCSD we also approximate the aplitude $C_{abcd}^{rstu}$
by products of the form "$C_{ab}^{rs} * C_{cd}^{tu}$". This has 8 different terms because there is many possibilities
of grouping $abcd$ and $rstu$ indices.
\end{solution}

\question Does variational-perturbation (p6 line 3) mean that reference wfns for perturbation theory are obtained using variational method?  Is correlation energy (or energy minimum) obtained usually using PT, whereas reference wfns for PT are obtained using variational method?  Otherwise, could you briefly explain it with example for me?

\begin{solution}
Frankly speaking, 
it is too advanced to explain in detail in this answer. I can only say generally
that variationally we can easily compute 1-st order electrostatic
interaction energy and exchange-repulsion energy. It also specifies the delocalization term. The "perturbational"
part of the method's name is because of 
the interpretation of the resulting terms on the grounds of the so called \emph{symmetry
adapted perturbation theory} which is the most laborious method developed for intermolecular interactions so far. 
Also, perturbational part of EDS method can be due to its extension to treat electron correlation
perturbatively by using MP2 method.
\end{solution}

\question Could you explain briefly how the reference wfns (MO) for perturbation theory 
are obtained by LCAO in the case of B3LYP/6-311++G** and CCSD/6-311++G**?

\begin{solution}
B3LYP is variational calculation so there is no PT involved there.
In CCSD (or in general, CC) the perturbation is in a cluster representation of the
correlation energy operator. I need to emphasize that CC is not RSPT type of theory and the treatment
is mathematically different. We were drawing the muptiply excited determinants recently (with singles, doubles etc).
You can view these functions as sort of reference basis. Perturbation theory in CC is based
on the following parameterization: $H \rightarrow e^{-\lambda \hat{T}} H_0 e^{\lambda \hat{T}}$ where
$\hat{T}$ is an excitation operator (related to the process we were doing recently by replacing MOs
in the reference (zeroth-order) wavefunction. Now, Taylor-expanding the above transformed Hamiltonian
with respect to $\lambda = 0$ (no perturbation - no excitations) we arrive at CC equations at various
orders wrt $\lambda$. It is more practical to refer to the type of excitations included rather than this order. So, we used
to write CCSD, CCSDT etc, not 3-rd order CC etc. However, in general, CC equations can be written so that
we define $\hat{T} = \hat{T}^{(1)} + \hat{T}^{(2)} + \ldots$. This leads to the following equation: 
$\lvert 0\rangle^{(n)} = \left[e^{\hat{T}}\right]^{(n)} \lvert HF \rangle$ where $n$ is the order
of CC.
We can say that $n$th-order excitations enter in $(n-1)$th order or CC perturbation theory.
For instance, The 3rd-order contains some type of quadruple excitations and corrections to singles, 
doubles and triples.
\end{solution}


\end{questions}


\end{document}
