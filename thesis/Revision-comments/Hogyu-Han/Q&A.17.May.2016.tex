\documentclass{exam}
\printanswers
\usepackage{color}
\shadedsolutions
\renewcommand{\solutiontitle}{\noindent\textbf{Answer:}\enspace}
\definecolor{SolutionColor}{rgb}{0.8,0.9,1}
%\framedsolutions

% graphics
\usepackage{graphicx}
\usepackage{wrapfig}
\usepackage{lipsum}
\usepackage{afterpage}

% tables
\usepackage{dcolumn}
\usepackage{multirow}
\usepackage{hhline}
\usepackage{booktabs}
%\setlength{\tabcolsep}{0.5em} % for the horizontal padding
%{\renewcommand{\arraystretch}{2.2}% for the vertical padding

% mathematics
\usepackage{amsmath}
\usepackage{amsfonts}
\usepackage{amssymb}
\usepackage{amsbsy}
\usepackage{bm}
\usepackage{mathrsfs}
\usepackage{upgreek}
\allowdisplaybreaks

% symbols
\usepackage{gensymb}
\usepackage{textcomp}

% footnotes
\usepackage[symbol]{footmisc}

% caption fonts
\usepackage[font=scriptsize,labelfont=bf]{caption}

% boxed multiline equations
\usepackage[overload]{empheq}

\newcommand*{\widebox}[2][0.5em]{\fbox{\hspace{#1}$\displaystyle #2$\hspace{#1}}}
\usepackage{xpatch}%
\makeatletter
\xpatchcmd{\@Aboxed}{%
\boxed {#1#2}}
{%
\color{HotPink3}\boxed {\color{textcolor}#1#2}}
{}{}
\makeatother

%---------------------------------------------------
% SHORTCUTS
\newcolumntype{,}{D{.}{,}{2}}
\newcommand{\citee}[1]{\ensuremath{\scriptsize^{\citenum{#1}}}}
\newcommand{\HRule}{\rule{\linewidth}{0.2mm}}
% Quantum notation
\newcommand{\bra}[1]{\ensuremath{\bigl\langle {#1} \bigl\lvert}}
\newcommand{\ket}[1]{\ensuremath{\bigr\rvert {#1} \bigr\rangle}}
\newcommand{\braket}[2]{\ensuremath{\bigl\langle {#1} \bigl\lvert {#2} \bigr\rangle}}
\newcommand{\tbraket}[3]{\ensuremath{\bigl\langle {#1} \bigl\lvert {#2} \bigl\lvert {#3} \bigr\rangle}}
% Math
\newcommand{\pd}{\ensuremath{\partial}}
\newcommand{\DR}{\ensuremath{{\rm d} {\bf r}}}
%\newcommand{\BM}[1]{\ensuremath{\mbox{\boldmath${#1}$}}}
\newcommand{\BM}[1]{\bm{#1}}
% Chemistry (formulas)
\newcommand{\ch}[2]{\ensuremath{\mathrm{#1}_{#2}}}
% Math 
\newcommand{\VEC}[1]{\ensuremath{\mathrm{\mathbf{#1}}}}
% vector nabla
\newcommand{\Nabla}{\ensuremath{ \BM{\nabla}}}
% derivative
\newcommand{\FDer}[3]{\ensuremath{
\bigg(
\frac{\partial #1}{\partial #2}
\bigg)_{#3}}}
% diagonal second derivative
\newcommand{\SDer}[3]{\ensuremath{
\biggl(
\frac{\partial^2 #1}{\partial #2^2}
\biggr)_{#3}}}
% off-diagonal second derivative
\newcommand{\SSDer}[4]{\ensuremath{
\biggl(
\frac{\partial^2 #1}{\partial #2 \partial #3}
\biggr)_{#4}}}
% derivatives without bound
% derivative
\newcommand{\fderiv}[2]{\ensuremath{
\frac{\partial #1}{\partial #2}}}
% diagonal second derivative
\newcommand{\sderiv}[2]{\ensuremath{
\frac{\partial^2 #1}{\partial #2^2}
}}
% off-diagonal second derivative
\newcommand{\sderivd}[3]{\ensuremath{
\frac{\partial^2 #1}{\partial #2 \partial #3}
}}
% derivatives for tables
\newcommand{\fderivm}[2]{\ensuremath{
{\partial #1}/{\partial #2}}}
% diagonal second derivative
\newcommand{\sderivm}[2]{\ensuremath{
{\partial^2 #1}/{\partial #2^2}
}}
% off-diagonal second derivative
\newcommand{\sderivdm}[3]{\ensuremath{
{\partial^2 #1}/{\partial #2 \partial #3}
}}

%############################################################################
\begin{document}

\section{p32 eq 3.10a \& eq 3.11}
\begin{questions}

%
\question Did you use eq 3.10a for RSPT to obtain a small set of full Hamiltonian, whereas eq 3.11 was used to obtain 
the full Hamiltionian and thereby Hessian matrix? Otherwise, could you explain the difference between 3.10a \& eq 3.11 
in the purpose and usuage of these two eqs? That is, why do you refer these two equations instead of either one of them?

\begin{solution}
Eqs. 3.10 (both) denote the general basis vectors whereas Eq. 3.11 describes the basis vectors formed from
uncoupled (or 'pure') states, i.e., where we assume all 1-D oscillators are completely independent from each other.
I use Eq. 3.11 further for derivations (to obtain energetic levels for both wavefunctions in 3.10a and 3.10b). 
I do not need to obtain explicit (full) Hamiltonians here. That is why I use RSPT. Otherwise, 
the solution would be provibitively troublesome.
\end{solution}

%
\question Can ket $\vert 1_0 2_0 \ldots j_n \ldots \rangle_{\Psi}$ in eq 3.10a be expressed as (or equal to) 
$\vert 1_0 \rangle \otimes \vert 2_0 \rangle \otimes \ldots \otimes \vert j_n \rangle \otimes \ldots$
The reason for this question is that i wonder whether eq 3.10a can be expressed using the cross product 
analogously to the cross product expression in eq 3.11.

\begin{solution}
No, it cannot. The reason is the coupling between 1-D oscillators (normal modes). The coupling exists even in gas phase.
Adding solvent molecules (solvating the solute) introduces even more couplings through $U$ -- the solute-solvent interaction potential.
The state $\vert 1_0 \rangle \otimes \vert 2_0 \rangle \otimes \ldots \otimes \vert j_n \rangle \otimes \ldots$ 
is only an idealized state made of completely uncoupled normal modes (fully harmonic approximation). We use them as the \emph{reference}
states in RSPT because we know exact solutions to the Schr{\"o}dinger equation under harmonic approximation 
(i.e., uncoupled harmonic oscillators).
\end{solution}

%
\question For eq 3.11, there is eq 3.12. Likewise, can there be an analogous eq or expression for 3.10?

\begin{solution}
Then it will be just an operator acting on state vector, i.e., $\hat{Q}_i \vert 1_0 2_0 \ldots j_n \ldots \rangle_{\Psi}$.
It cannot be simplified any further because we did not specify formally how the state $\vert 1_0 2_0 \ldots j_n \ldots \rangle_{\Psi}$
is constructed. It can be any state in a Hilbert space spanned by eigenstates of a given Hamiltonian.
\end{solution}

%
\question Coud you explain why Fock is referred in the terminology "composite Fock"? 
That is, are there other composite spaces? If so, could you give me one example and explaination 
for the difference between Fock and other composite spaces?

\begin{solution}
I used the term \emph{Fock space} to maintain generality. Fock space is a construct that is designed to describe the Hilbert spaces
of a collection of indistinguishable particles (or excitations). 
In the Thesis, we deal with only single excitations and perhaps I could just
omit "Fock" and specify it as just "composite" space.
The term \emph{composite} means that the Hilbert space is obtained by 
an outer product of the sub-spaces. Here, sub-spaces refer to each normal mode space. So the output is the composite space
of (pure) eigenstates of (uncoupled) normal mode harmonic Hamiltonian: 
$\mathscr{H} = \sum_i \left\{ \frac{\hat{P}_i^2}{2M_i} + \frac{1}{2}M_i\omega_i^2 \hat{Q}_i^2 \right\}$.
\end{solution}

%
\question For three normal modes and four energy levels in the five 1D oscillators, 
how many vectors are cantained in the composite Fock space expressed by eq 3.12? 
This might be an awkward question. That is, if five 1D oscillators automatically 
mean five normal modes, whereby it conflicts with my question for three normal modes, 
could you answer my question for three normal modes and four energy levels in the separate 1D oscillator instead?

\begin{solution}
Did you mean Eq.3.11? Eq.3.12 specifies the composite operator, not state vector. 
Also, three normal modes mean three 1D harmonic oscillators. So I will consider 
the case of three normal modes and 4 energy levels per each normal mode (ground + 3 excited states). 
It means there is $4^3=64$ distinct
$\vert S \rangle$ states. It is a lot of basis vectors, producing Hamiltonian of the size $64\times 64$.
\end{solution}

%
\question What is the Hamiltonian matrix size (? by ?) if it is constructed 
from the composite space $\vert S \rangle$ in eq 3.11 given the above question 1-5?

\begin{solution}
The answer is given above. The question does not make sense for Eq.3.12 because it describes the operator.
\end{solution}

%
\question In your calculation for thesis, how many number of normal modes, energy levels, 
and 1D oscillators did you use? Could you set these number for calculation? Or, does computer set these number automatically?

\begin{solution}
It depends on the molecule. The number of modes for non-linear molecules is equal to 
$3N-6$ where $N$ is the number of atoms.
The number of harmonic (1D) oscillators is equal to the number of modes. 
We are interested only in the fundamental transitions, so 1$\leftarrow$0 transitions. 
In the calculations, I do not need to consider constructions of Hamiltonian in basis of $\vert S \rangle$ vectors
because we use RSPT.
\end{solution}


\end{questions}

%%%%%%%%%%%%%%
\section{p40 line between eqs 3.59 and 3.60.}

\begin{questions}
\question Could you explain the meaning of "traceless"?

\begin{solution}
"Traceless" means that the trace of a given tensor (or operator) vanishes: ${\rm Tr}\; {\bf q} \equiv \sum_i q_{ii} = 0$.
Trace is independent on the representation (basis set) for QM operators.
\end{solution}

\question What is the word "distributed" meant for in front of "traceless"? 
Ares there nondistributed moments? If so, what is it? 

\begin{solution}
Using the term "distributed moment" denotes the we divide the total multipole moment of the molecule
into parts describing some fragments of charge density distributions. For example, the distribution
can be done in such a way the resulting \emph{distributed} multipoles
are centered on atoms. The summation over $x$ is signalling about the use of distributed moments.
If we wanted to use total multipoles the summation would be omitted. The term "nondistributed" 
is not proper since any multipole is always
centered at some point.
\end{solution}


\end{questions}

%%%%%%%%%%
\section{p40 eqs 3.60 \& 3.61}

\begin{questions}
\question Are there any ways to define quadrupole and octupole moments other than eqs 3.60 \& 3.61? 
Is the traceless cartesian moments always defined as eqs 3.60 \& 3.61?

\begin{solution}
Yes - one can define cartesian traceless multipole moments basically in many different ways. Here, we use Cartesian
multipoles but it is more convenient to use spherical tensor notation which is unique. Starting from spherical tensor
notation it is then easier to notice the way of how we can define cartesian traceless multipole moment.
However, the freedom is perhaps only in the multiplying factor. For example, Jackson defines traceless
quadrupole and octupole as in Eq.3.60 and 3.61 but without the factors of $\frac{1}{2}$. This results in different
numerical coefficients in the series expansion of the electrostatic potential. See Jackson, J. D.,
\emph{Classical Electrodynamics}, third ed., Chapter 4 for more detailed discussion.
\end{solution}

\end{questions}

%%%%%%%%%%
\section{p56 sec 4.2 Title}

\begin{questions}
\question Could you explain the meaning of "coarse-graining"? What is other than coarse-graining?

\begin{solution}
Coarse-graining means the proces of \emph{reduction} in the true (complicated) system.
It results in a set of simpler fragments that can interact with each other and in this way
approximately describe the original system. Here, we use it to transform from the intricate atomic granularity
to the set of molecule-like objects that interact with each other. This is as opposed to implicit models
in which we treat the system as continuum. If we didn't want to use coarse-graining and describe the real system
we need to solve Schr{\"o}dinger equation for the whole set of atoms constructing the solute-solvent environment.
This is prohibitively difficult.
\end{solution}

\end{questions}

%%%%%%%%%%
\section{p32 eq 3.13}

\begin{questions}
\question Is notation "summation over k, l, ...n" omitted in front of 2nd equation 
(that is, expressed in cross products) in eq 3.13?

\begin{solution}
No - this summation is just a sum over all possibilities of normal mode permutations.
\end{solution}

\question Is notation "summation over j" also omitted in front of 2nd equation (that is, expressed in cross products) in eq 3.13?

\begin{solution}
No. But there is an incorrect choice of indice labels here. I should use other indices so that $j$ do not clash with each other.
I will use $J$ for $G_{n,J}$.
\end{solution}

\question Is notation "summation over k, l, ...n" (and "summation over j"?) 
used only when index notation rather than cross product notation (eq 3.11 uses the cross product notation) is used?

\begin{solution}
In this example yes. I don't know what about other equations since there is no rule here. But generally
I explicitly sum over all modes because it is difficult (or even impossible) to denote that by using
dot notation.
\end{solution}

\question Why is $\vert 1_k \rangle \otimes \vert 2_l \rangle \ldots \vert j_n \ldots \rangle $
rather than $\vert 1_0 \rangle \otimes \vert 2_0 \rangle \ldots \vert j_n \ldots \rangle $ ? 
The reason for this question is that I wonder why the restriction for eq 3.10 is not for eq 3.11?

\begin{solution}
Because $\vert S \rangle$ from Eq.3.11 is a set of reference vectors for RSPT calculations and they are not real.
It is safe to sum over all possibilities when evaluating properties by RSPT.
States from Eq.3.10 refer to real situations. We want to consider only some subset of real states, thus we
restrict them in a way provided in the text.
\end{solution}

\end{questions}

%%%%%%%%%%
\section{1. p63 eq 4.52, p71 eq 4.90}

\begin{questions}
\question You gave me the answer for derivation of these equations already.
But the notations written directly on the e-mail is hard to read, even though I can understand your answer in the case of eq 4.90.
May I ask your answer again for derivation of these two equations, 
but they are written separately in LaTeX and sent as pdf file in this time?
In particular, could I ask for the detailed derivation of eq 4.52? 
This is because I still have a difficulty in deriving eq 4.52 on the basis of your answer. 
I feel sorry for asking you the tedious work like this.

\begin{solution}
It is no problem at all. 

We have
%
\begin{equation}
r_{ab} = \sqrt{ [x_a-x_b]^2 + [y_a-y_b]^2 + [z_a-z_b]^2 }
\end{equation}
%
To compute derivative of $r_{ab}$ wrt $Q$ we can
explicitely differentiate the above equation. We can notice that
%
\begin{equation}
\frac{\partial}{\partial Q} \left( \sqrt {f(x,y,z)} \right) = \left( \frac{1}{2 \sqrt{f(x,y,z)}} \right)  \fderiv{f(x,y,z)}{Q}
\end{equation}
%
Above I abbreviated the set $(x_a, x_b, y_a, y_b, z_a, z_b)$ by $(x,y,z)$.
The function $f(x,y,z) = [x_a-x_b]^2 + [y_a-y_b]^2 + [z_a-z_b]^2$.

Now, we see that $\sqrt{f(x,y,z)} = r_{ab}$. By computing the derivatives $\fderiv{f(x,y,z)}{Q}$ we get
%
\begin{equation}
\fderiv{f(x,y,z)}{Q} = 2[x_a-x_b]  \left(\fderiv{x_a}{Q} - \fderiv{x_b}{Q}\right)
                     + 2[y_a-y_b]  \left(\fderiv{y_a}{Q} - \fderiv{y_b}{Q}\right) 
                     + 2[z_a-z_b]  \left(\fderiv{z_a}{Q} - \fderiv{z_b}{Q}\right)
\end{equation}
%
Note that the above result can be written in a dot product form of two vectors:
%
\begin{equation}
\fderiv{f(x,y,z)}{Q} = 2 [ {\bf r}_a - {\bf r}_b ] \cdot \left(\fderiv{{\bf r}_a}{Q} - \fderiv{{\bf r}_b}{Q}\right)
\end{equation}
%
Thus, finally we get the result:
%
\begin{equation}
\fderiv{f(x,y,z)}{Q} = \frac{1}{2r_{ab}} 
   \times 2 [ {\bf r}_a - {\bf r}_b ] \cdot \left(\fderiv{{\bf r}_a}{Q} - \fderiv{{\bf r}_b}{Q}\right)
 = r_{ab}^{-1} [ {\bf r}_a - {\bf r}_b ] \cdot \left(\fderiv{{\bf r}_a}{Q} - \fderiv{{\bf r}_b}{Q}\right)
\end{equation}
%d\dQ (sqrt (f(x,y,z)) = (0.5 / r_{ab} ) * 2 [ r_a - r_b ] · (dr_a/dQ - dr_b/dQ) = r_{ab}^{-1} *  [ r_a - r_b ] · (dr_a/dQ - dr_b/dQ)
which is Eq. 4.90.

Now, let us derive Eq.4.52. Here I switch indices $x\rightarrow a$ and $y\rightarrow b$ 
to be consistent with the above example. 

It is much easier if first we will figure out the following
derivative: $\fderiv{r^{-n}}{Q}$ for $n=1,2,\ldots$. We have
%
\begin{equation}
\fderiv{r_{ab}^{-n}}{Q} = -n r_{ab}^{-(n+1)} \times \fderiv{r_{ab}}{Q}
\end{equation}
%
Note that we already got the result for $\fderiv{r_{ab}}{Q}$ earlier. Therefore we can write
%
\begin{equation}
\fderiv{r_{ab}^{-n}}{Q} = -n r_{ab}^{-(n+2)} [ {\bf r}_a - {\bf r}_b ] \cdot \left(\fderiv{{\bf r}_a}{Q} - \fderiv{{\bf r}_b}{Q}\right)
\end{equation}
%
Now, the dipole-dipole interaction tensor can be written as
%
\begin{equation}
 T_{\alpha\beta} = 3\frac{r_\alpha r_\beta}{r^5} - \frac{\delta_{\alpha\beta}}{r^3}
\end{equation}
%
where, for brevity, I used the short-cut notation $r_\alpha \equiv {\bf r}_{ab;\alpha}$ (i.e., the $\alpha$th Cartesian component
of the ${\bf r}_{ab}$ vector. Also, $r^n \equiv \vert {\bf r}_{ab} \vert^n $.

Taking the derivative wrt $Q$ we have
%
\begin{equation}
 \fderiv{T_{\alpha\beta}}{Q} = 
 3r_\alpha r_\beta \fderiv{r^{-5}}{Q} + 
 3r^{-5} \left[ \fderiv{r_\alpha}{Q} r_\beta  + r_\alpha \fderiv{r_\beta}{Q} \right]
    - \delta_{\alpha\beta} \fderiv{r^{-3}}{Q}
\end{equation}
%
%In the above result, I already assumed that $\fderiv{{\bf r}_b}{Q}=0$ because $b$ label
%refer to the solvent center. 
Now, using our above result we can write that
%
\begin{equation}
 \fderiv{r^{-5}}{Q} = -5r^{-7} [ {\bf r}_a - {\bf r}_b ] \cdot \left(\fderiv{{\bf r}_a}{Q} - \fderiv{{\bf r}_b}{Q}\right)
\end{equation}
%
Note that $\fderiv{{\bf r}_b}{Q}\cong 0$ because $b$ label
refer to the solvent center while $Q$ is solute's normal mode. 
Therefore the above result can be simplified:
%
\begin{equation}
 \fderiv{r^{-5}}{Q} = -5r^{-7} [ {\bf r}_a - {\bf r}_b ] \cdot \fderiv{{\bf r}_a}{Q} 
\end{equation}
%
Analogously,
%
\begin{equation}
 \fderiv{r^{-3}}{Q} = -3r^{-5} [ {\bf r}_a - {\bf r}_b ] \cdot \fderiv{{\bf r}_a}{Q} 
\end{equation}
%
We now try to put all these results together. The outcome is
%
\begin{equation}
 \fderiv{T_{\alpha\beta}}{Q} = 3 r^{-5} [ {\bf r}_a - {\bf r}_b ] \cdot \fderiv{{\bf r}_a}{Q} 
 \left\{ 
   \delta_{\alpha\beta} - 5 r_\alpha r_\beta r^{-2}
 \right\}
 + 
 3r^{-5} \left[ \fderiv{r_\alpha}{Q} r_\beta  + r_\alpha \fderiv{r_\beta}{Q} \right]
\end{equation}
%
Now, the last step is to switch to the "bold" notation and use "$\otimes$"
operator. By doing that we notice that
%
\begin{subequations}
\begin{align}
 r_\alpha r_\beta                &\rightarrow {\bf r}_{ab} \otimes {\bf r}_{ab} \\
 r_\alpha \fderiv{r_\beta}{Q}    &\rightarrow {\bf r}_{ab} \otimes \fderiv{{\bf r}_{ab}}{Q} \\
 \fderiv{r_\alpha}{Q} r_\beta    &\rightarrow \fderiv{{\bf r}_{ab}}{Q} \otimes {\bf r}_{ab} \\
 \delta_{\alpha\beta}            &\rightarrow {\bf 1}
\end{align}
\end{subequations}
%
when switching in between the two "index" and "dot-cross" or "bold" notations.
Therefore, the derivative of $T_{\alpha\beta}$ written in "bold" notation
reads
%
\begin{equation}
 \fderiv{{\bf T}}{Q} = 3 r^{-5} [ {\bf r}_a - {\bf r}_b ] \cdot \fderiv{{\bf r}_a}{Q} 
 \left\{ 
   {\bf 1} - 5 r^{-2} {\bf r}_{ab} \otimes {\bf r}_{ab}
 \right\}
 + 
 3r^{-5} \left[ \fderiv{{\bf r}_{ab}}{Q} \otimes {\bf r}_{ab}  + {\bf r}_{ab} \otimes \fderiv{{\bf r}_{ab}}{Q} \right]
\end{equation}
%
which is identical to Eq.4.52 except for the power over $r$. 
In the Thesis draft there appears to be 3 - it should be corrected to 5 and
it is a typo. To make sure, I checked the code and there is a correct $r^{-5}$
power.

\end{solution}

\end{questions}

%%%%%%%%%%
\section{p190 table B.1}

\begin{questions}
\question You already gave me the answer for how to derive Ui and Ujj terms for $r^{-2}$.
Likewise as above, the notations written directly on the e-mail is hard to read.
Thus, may I ask again your answer written in LaTeX?
Then, I will try to derive Ui and Ujj terms for $r^{-3}$.

\begin{solution}
Sure.

From Eq.4.32a we have that
%
\begin{equation}
U_i = q_x   {\bf L}_x^{(i)} \cdot \nabla \phi_x 
\end{equation}
%
From Eq.4.33a we have that
%
\begin{equation}
U_{jj} = 2\sderiv{q_x}{Q_j}  {\bf L}_x^{(i)} \cdot \nabla \phi_x  + \text{ term containing gradient of gradient of $\phi$ at $x$}
\end{equation}
%
We need to work out the forms of electrostatic potential gradients. 
They are specified in Eq. 4.27. We can notice that
%
\begin{equation}
\nabla \phi_x = - \sum_y    q_y  \frac{r_{xy}}{ r_{xy}^3} \text{ (this is $r^{-2}$ term) + $r^{-3}$ and higher-order terms}
\end{equation}
%
Analogously we notice that $\nabla \phi_x$ (gradient of gradient of $\phi$ at $x$) contains only $r^{-3}$ 
and higher-order terms contributions. Thus, we do not need to consider it for $r^{-2}$ terms.

Let us now evaluate $U_i$ for $r^{-2}$ by using the above result for $\nabla \phi_x$.
%
\begin{equation}
U_i = -q_x q_y {\bf L}_x^{(i)} \cdot  {\bf r}_{xy}   r_{xy}^{-3}  
\end{equation}
%
(I have dropped summation over $y$ since we count here contribution for a $x,y$ pair).
This expression is the result given in Table B.1. Analogously, we get the $U_{jj}$ term for $r_{xy}^{-3}$.

In general, the equations from Table B.1 for $r_{xy}^{-3}$ and $r_{xy}^{-4}$ are extremely tedious in derivation. 
Moreover, the derivative wrt $Q$ of multipole series expansion does not need to converge well 
even if the source multipole series does. That is why I terminate the electrostatic 
potential correction term series on $r_{xy}^{-2}$ contributions to estimate roughly these effects 
that are not inlcuded in SolCAMM method.
\end{solution}

\end{questions}


\end{document}
#############################################################################


\section{Differential Equations}
\begin{questions}
\question ... Short question here ... \answerline[Short answer]
\question ... Question here ... 

\begin{solution}[.2in]
... Answer here ... 
\end{solution}

\question Long descriptive question about everything

\begin{solutionorlines}[2in]
Long descriptive answer is a long descriptive answer that is a long descriptive answer that is along descriptive answer that is along descriptive answer that is along descriptive answer that is a lie.
\end{solutionorlines}

\question Draw an arrow showing north direction.

\begin{solutionorbox}[2in]
$\uparrow$
\end{solutionorbox}

\end{questions}

\ifprintanswers
Stuff to appear only when answers \textbf{are} being printed.
\else
Stuff to appear only when answers \textbf{are not} being printed.
\fi
